\documentclass[]{article}
\usepackage{amsmath}
\usepackage{amsfonts}
\usepackage{amssymb}

\usepackage{showlabels} % debug, togliere

%\usepackage[italian]{babel}
%\usepackage[latin1]{inputenc}
%\usepackage[utf8x]{inputenc}
\usepackage{dsfont}
%\usepackage{amsthm}
%\usepackage[cm]{fullpage}
\usepackage{enumerate}
%\usepackage{extarrows}
%\usepackage{mathrsfs}
%\usepackage{braket}
%\usepackage{wrapfig}
\usepackage{tikz}
%\usepackage{verbatim}

\usepackage{hyperref}%deve essere l'ultimo package
%http://www.tug.org/applications/hyperref/manual.html
\hypersetup{colorlinks=true,
linkcolor=black}

%derivate
\newcommand{\de}[2]{\ensuremath{\frac{\mathrm{d} #1}{\mathrm{d} #2}}}
\newcommand{\lde}[2]{\ensuremath{{\mathrm{d} #1}/{\mathrm{d} #2}}}
\newcommand{\dt}[2]{\ensuremath{\frac{\mathrm{d} #1}{\mathrm{d} t}}}
\newcommand{\Dt}[1]{\ensuremath{\frac{\mathrm{D} #1}{\mathrm{D} t}}}
\newcommand{\pde}[2]{\ensuremath{\frac{\partial #1}{\partial #2}}}
\newcommand{\lpde}[2]{\ensuremath{{\partial #1}/{\partial #2}}}
%lettere
\newcommand{\df}{\ensuremath{\mathrm{d}}}
\newcommand{\w}{\ensuremath{\omega}}
\newcommand{\T}{\ensuremath{\Theta}}
\newcommand{\R}{\mathds{R}}
\renewcommand{\O}{\mathcal{O}} %ordine di
%operatori
\newcommand{\Var}{\text{Var}}
%parentesi
\newcommand{\lr}[3]{\ensuremath{\left#1 #3 \right#2}}
\newcommand{\lrt}[1]{\lr{(}{)}{#1}}
\newcommand{\lrq}[1]{\lr{[}{]}{#1}}
\newcommand{\lrg}[1]{\lr{\{}{\}}{#1}}
\newcommand{\media}[1]{\lr{<}{>}{#1}}

\newcommand{\infint}{\int\limits_{-\infty}^{+\infty}}
\newcommand{\MB}{Maxwell-Boltzmann}

%vettori
\renewcommand{\vec}[1]{\boldsymbol{#1}}

\numberwithin{equation}{subsection}

%opening
\title{Appunti di Sistemi Mesoscopici e Nanodispositivi}
\author{Daniele Rapetti}
\date{}

%\makeindex

\begin{document}
	\maketitle
	\section{Eterostrutture}
	Un-eterostruttura \`e una giunzione tra due materiali differenti. Si presenta il problema di cercare due materiale con strutture cristalline il pi\`u simili possibile per eveitare stress alla giunzione e per rispettare il pi\`u possibile il teorema di Bloch.
	
	Per esempio Semiconduttori simili sono
	\begin{itemize}
		\item GaAs e AlAs
		\item GaIn e e InP
	\end{itemize}
	In quanto hanno una costante cristallina molto simile.
	
	A questo punto scegliamo la prima coppia, in quanto GaAs ha il gap diretto e ci dedicheremo allo studio della giunzione 
	\begin{equation*}
	\text{GaAs}\leftrightarrow\text{Al}_x\text{Ga}_{1-x}As
	\end{equation*}
	
	la $x$ indica la $\%$ di Al rispetto al Ga nel secondo mezzo, varia tra 0 e 1, solitamente si scelgono valori inferiori a 0.4 in quanto a 0.45 il gap diventa indiretto.
	
	\begin{figure}[h]
		\caption{}
		immagine dei due livelli dei materiali vicini $\Delta E_c = E_{c2}-E_{c1} = 0.77 x$ e $\Delta E_v = E_{c1}-E_{c2} = 0.48 x$
	\end{figure}
	
	A questo punto droghiamo: il lato a destra del sistema con donori e la part senza alluminio con accettori e procediamo con i calcoli della giuntura unita. Il drogaggio sar\`a tale che i donori saranno molti pi\`u degli accettori:
	\begin{equation*}
	N_A\ll N_D  \Rightarrow n \gg p
	\end{equation*}

	Definisco la differenza tra le due energie di Fermi:
	\begin{equation}
	E_{F1} - E_{F2} = e\phi_0
	\end{equation}
	
	\begin{figure}
		immagine lacune e e- in movimento
	\end{figure}
	Ho quindi movimento di lacune e elettroni, il che mi dar\`a luogo  a una corrente di deriva:
	
	\begin{equation}
	J_{d}^{(n)}  = e D_n \de nx
	\end{equation}
	
	\begin{figure}
		immagine distribuzioni di carica
	\end{figure}
	
	La rilocazione di lacune e elettroni crea una distribuzione di carica non omogenea che da luogo a un campo elettrico che si oppone alla corrente di deriva. 
	\begin{equation}
	E(z) = -\de{\phi}z, \ \ \de{^2\phi}{z^2} = -\frac{\rho_i}{\varepsilon_0 \varepsilon_{r i}}
	\end{equation}
	dove $i$ indica il mezzo 1 o 2. D'ora in poi sintetizzer\`o:  $\varepsilon_0 \varepsilon_{r i} = \varepsilon_i$
	
	Definisco le condizioni al contorno del campo:
	\begin{eqnarray}
	\phi(0^+) = 	\phi(0^-)\\
	\varepsilon_1 \phi'(0^+) = 	\varepsilon_2 \phi'(0^-)\\
	\phi(z) = costante & \text{ se } z \geq \df n\text{ o } z \leq -\df p
	\end{eqnarray}
	\begin{equation}
	\phi(+\infty) - \phi(-\infty) = \phi_0
\end{equation}

	\begin{figure}
		\caption{Il potenziale cambia la posizione delle bande}
	\end{figure}
	e avremmo 
	\begin{equation}
	\phi(z) = \lr\{ . {\begin{array}{lr}
\phi_1 + e \frac{N_a}{2\varepsilon_1}\lrt{z+\df p}^2 & -\df p \leq z \leq 0\\
\phi_2 + e \frac{N_a}{2\varepsilon_2}\lrt{z-\df n}^2 & 0 \leq z \leq \df n
		\end{array}}
	\end{equation}
	Per comodit\`a definiamo $\phi_2 = 0$, come ipotesi \`e ragionevole perch\`e nel mezzo 2 il drogaggio \`e molto intenso e quindi \`e in condizione di essere ''dominante'' sul mezzo 1. In pratica sto schematizzando che l'$E_{F1}$ si sposter\`a molto meno rispetto all'$E_{F1}$.
	
	Rispetto al caso di una giunzione omogenea le dimensioni delel zone di svuotamento si calcolano:
	\begin{equation}
	\df n = \lrt{\frac{N_a}{N_d} \frac{1}{\varepsilon_1 N_a + \varepsilon_2 N_d} \frac{2\varepsilon_1 \varepsilon 2 \phi_0 }{e}}^{\frac 12}
	\end{equation}
		\begin{equation}
		\df p = \lrt{\frac{N_d}{N_a} \frac{1}{\varepsilon_1 N_a + \varepsilon_2 N_d} \frac{2\varepsilon_1 \varepsilon 2 \phi_0 }{e}}^{\frac 12}
		\end{equation}
		\begin{figure}
			Immagine risultato
		\end{figure}
\end{document}
